Je vais présenter dans cette partie \emph{mediarithmics} ainsi que son activité. Dans un premier temps je vais 
aborder la création de \emph{mediarithmics} et son développement commercial. Puis, dans un second temps je 
présenterai la thématique du data marketing dans lequel son activité s'inscrit.
    \section{Historique et développement de l'entreprise}
        \subsection{Développement}
            Fondée en 2012 par Stephane \textsc{Dugelay}, \emph{mediarithmics} est une plate-forme ouverte de 
            \textbf{data marketing}. Volontairement ouverte, la plate forme \text{SaaS}\footnote{Software as a Service} 
            \emph{mediarithmics} fournit a ses clients un ensemble d'outils lui permettant de gérer ses campagnes 
            marketing de bout en bout, du stockage de données utilisateur au déploiement de campagnes publicitaires 
            personnalisées. Après plusieurs années de recherche \& développement, \emph{mediarithmics} a débute son 
            activité commerciale en 2015 et représente maintenant un acteur majeur du \emph{digital marketing} français. 
            Fort de son expertise nationale notamment dans le cadre des alliances média avec l'alliance 
            \emph{Gravity}\footnote{\url{https://fr.wikipedia.org/wiki/Alliance\_Gravity}} \emph{mediarithmics} s'ouvre 
            maintenant les portes du marche international afin de continuer son développement.
        \subsection{Organisation}
            Dans cette sous-partie je vais seulement décrire l'organisation au de l'équipe de recherche et développement.
            En tant que jeune entreprise en développement, l'organisation de l'équipe est très horizontale et simple. 
            En effet à mon arrivée, Stéphane \textsc{Dugelay}, avec l'aide des responsables de produits s'occupaient de 
            définir la feuille de route du produit ainsi que sa direction. Puis, les tâches étaient attribuées aux 
            développeurs de l'équipe. Récemment, l'équipe grandissant, ce fonctionnement a montré ses limites. En effet 
            ce système peut entraîner un manque de lisibilité sur l'avancement global du produit pour les développeurs. 
            De plus dans ce système il est difficile pour une équipe de mesurer sa vélocité. Au cours de mon stage, 
            cette organisation a commencé à évoluer grâce au recrutement d'un responsable de l'équipe et par la même 
            occasion la mise en place d'une structure agile plus lisible de type \emph{SCRUM}. Dorénavant l'ensemble de l'équipe est 
            répartie en \emph{squads} \cite{spotify}, avec la réalisation de sprints de deux semaines.

        \subsection{Localisation}
        % localisation
    \section{Contexte du \emph{data marketing}}
        \subsection{Qu'est ce que le data marketing ?}
        Le \emph{data marketing} consiste en l'utilisation de données clients en vue de réaliser des actions marketing. 
        Dans le contexte numérique (\emph{digital marketing}) elle représente deux activites principales: la collecte de
        données d'internautes et l'utilisation de ces dernières a des fins mercatiques. Comme tout secteur spécialisé, 
        le data marketing dispose de ses propres acronymes. Je vais en présenter deux qui représentent le coeur de l'activité
        de mediarithmics.
        \begin{itemize}
            \item DMP: une \textbf{Data Management Platform} est une plate-forme de gestion des données. Une DMP permet
            de collecter, stocker, normaliser et des données utilisateurs en provenance de divers supports: applications
            web, applications mobiles, courriels... Elle permet par exemple a un annonceur de créer des segments 
            d'utilisateurs par la suite utilises lors des campagnes marketing.
            \item DSP: une \textbf{Demand Side Platform} est une plate-forme technologique permettant la mise en relation
            entre les \emph{Adexchanges}\footnote{Place de marche virtuelle ou sont mis en vente les espaces publicitaires} et la donnée.
        \end{itemize}
        Développeur de \textbf{DMP} et \textbf{DSP}, \emph{mediarithmics} offre la possibilité a ses clients d'utiliser 
        l'une ou/et l'autre de ces fonctionnalités. En effet, un client peut, par l'intermédiaire de la 
        \textbf{DMP}, récolter et structurer de la donnée sur ses utilisateurs pour analyser son activité. 
        Puis grâce a cette DMP et aux informations sur les utilisateurs, le client peut ensuite l'utiliser pour 
        effectuer des campagnes marketing personnalisées.