\newglossaryentry{startup}
{
	name=start-up,
	description={Une startup, ou jeune pousse, est une jeune entreprise innovante à fort potentiel de croissance qui fait souvent l'objet de levées de fonds.}
}

\newglossaryentry{stereo}
{
	name=stéréo-vision,
	description={La stéréovision ou mesure stéréoscopique est une méthode de mesure qui consiste à se servir de la prise d'images (photographiques ou numériques) prises de différents points de vue, pour déterminer les dimensions, les formes ou les positions d'objets.}
}

\newglossaryentry{sdk}
{
	name=SDK,
	description={SDK est l'acronyme anglais pour Software Development Kit. Un SDK est un ensemble d'outils d'aide à la programmation proposé aux éditeurs / développeurs d'applications.}
}

\newglossaryentry{wrapper}
{
	name=wrapper,
	description={En programmation informatique, une fonction wrapper est un programme dont la fonction principale est d'appeler une autre fonction. Dans notre cas cela correspond au plugin entier permettant l'appel des différentes fonctions à utiliser.}
}


\newglossaryentry{baseline}
{
	name=baseline,
	description={La baseline est la distance maximale séparant les différents instruments de mesure, ici ce sont donc les caméras.}
}

\newglossaryentry{ouverture}
{
	name=ouverture,
	description={L'ouverture d'un objectif photographique est le réglage qui permet d'ajuster le diamètre d'ouverture du diaphragme.}
}


\newglossaryentry{frame-rate}
{
	name=frame-rate,
	description={Le frame-rate ou encore fps est le nombre d'images par seconde (ou images à la seconde) est une unité de mesure correspondant au nombre d'images affichées en une seconde par un dispositif.}
}

\newglossaryentry{fps}
{
	name=fps,
	description={Frame per second. Voir frame-rate.}
}

\newglossaryentry{SLAM}
{
	name=SLAM,
	description={La localisation et cartographie simultanées, connue en anglais sous le nom de SLAM (simultaneous localization and mapping) ou CML (concurrent mapping and localization), consiste, pour un robot ou véhicule autonome, à simultanément construire ou améliorer une carte de son environnement et de s’y localiser.}
}

\newglossaryentry{odometrie}
{
	name=odométrie,
	description={L’odométrie est une technique permettant d'estimer la position d'un véhicule en mouvement. }
}

\newglossaryentry{Mesh}{name={mesh},description={Le Mesh correspond à une forme géométrique et dans notre cas aux informations à propos de celle-ci.}}

\newglossaryentry{texture}
{
	name=texture,
	description={Une texture en traitement d'image est une série de mesures calculée dans le but de détecter une texture perçue sur une image.}
}


\newglossaryentry{runtime}
{
	name=runtime,
	description={Un environnement d'exécution ou runtime est un logiciel responsable de l'exécution des programmes informatiques écrits dans un langage de programmation donné. Ici le runtime correspond donc au lancement des différents processus pour certaines opérations réalisées pour la caméra.}
}

\newglossaryentry{mapping}
{
	name=mapping spatial,
	description={Le mapping spatial est ici une fonctionnalité permettant de créer une représentation 3D de l'environnement spatial de la caméra sous forme de carte.}
}

\newglossaryentry{Depth}
{
	name=Depth,
	description={La Depth est ici l'ensemble des informations sur la profondeur des éléments filmés par la caméra.}
}

\newglossaryentry{Point}
{
	name=Point,
	description={Le Point Cloud correspond ici à l'ensemble des données de profondeur sous la forme d'un nuage de points. Voir Depth et mapping.}
}

\newglossaryentry{Case}{name={Snake Case et Camel Case},description={Snake Case est une convention typographique en informatique consistant à écrire des ensembles de mots en minuscules en les séparant par des tirets bas. Cette convention s'oppose par exemple au Camel Case qui consiste à mettre en majuscule les premières lettres de chaque mot.}}


\newglossaryentry{CPU}
{
	name=CPU,
	description={Un processeur (ou unité centrale de traitement, UCT, en anglais central processing unit, CPU) est un composant présent dans de nombreux dispositifs électroniques qui exécute les instructions machine des programmes informatiques.}
}

\newglossaryentry{GPU}
{
	name=GPU,
	description={Un processeur graphique, ou GPU (de l'anglais Graphics Processing Unit), est un circuit intégré la plupart du temps présent sur une carte graphique (mais pouvant aussi être intégrée sur une carte-mère ou dans un CPU) et assurant les fonctions de calcul de l'affichage.}
}

\newglossaryentry{tracking}
{
	name=tracking,
	description={Le match moving est une technique utilisée dans le domaine des effets spéciaux et liée à la capture de mouvement. }
}

\newglossaryentry{Chunks}
{
	name=Chunks,
	description={Sous-partie du Mesh. Voir Mesh.}
}

